\documentclass{article}
\pagestyle{empty}
\usepackage{amsmath,amssymb}
\usepackage{url}
\usepackage{hyperref}

%% Ideally, I should make this into a file.
%% But I don't remember how to include the file :-)
\newcommand{\OR}{ \vee }
\newcommand{\AND}{ \wedge }
\newcommand{\IMPL}{ \rightarrow }
\newcommand{\EQUIV}{ \leftrightarrow }
\newcommand{\LOGEQ}{<\equiv> }
\newcommand{\NOT}{\widetilde{}\enspace}
\newcommand{\lam}{\lambda}

\begin{document}

%{\large\bf Work in pairs}



\begin{center}
{\large\bf Introduction to the $\lam$-calculus for CSci 4651}
\\ 
{\large\bf Elena Machkasova, UMN Morris}
\end{center}

\section{Overview}
\textit{The Lambda Calculus ($\lam$-calculus)} was developed by an American mathematician Alonzo Church in 1930s to study computation. 
It uses anonymous functions and their composition to represent computation. The $\lam$-calculus has been proven equivalent to 
the Turning Machine model of computation (invented by Alan Turing in 1936) which includes an infinite tape (modeling computer memory) and a collection of states of the computing device that perform different actions (modeling a program). 

The $\lam$-calculus and its more modern variants and extensions is used to formally define evaluation of a program, and 
specifically different orders of evaluation, including \textit{call-by-value} and \textit{call-by-name} evaluation strategies that we cover below, 
as well as \textit{call-by-need} that represents \href{https://en.wikipedia.org/wiki/Lazy_evaluation}{lazy evaluation strategy}, 
\textit{$\pi$-calculus (pi-calculus)} that represents a distributed system of processes communicating via channels, and many other. 

\section{Syntax}

$\lam$-calculus operates with expressions known as $\lam$-terms that we define below. 
Even though the classical $\lam$-calculus doesn't directly include numbers, it is possible to define $\lam$-terms 
that correspond to numbers and define
arithmetic operations on them. For simplicity we include integer constants directly into expressions in the 


\end{document}
